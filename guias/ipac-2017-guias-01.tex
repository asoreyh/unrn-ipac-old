\documentclass[a4paper,12pt]{article}
\usepackage[spanish]{babel}
\hyphenation{co-rres-pon-dien-te}
%\usepackage[latin1]{inputenc}
\usepackage[utf8]{inputenc}
\usepackage[T1]{fontenc}
\usepackage{graphicx}
\usepackage[pdftex,colorlinks=true, pdfstartview=FitH, linkcolor=blue,
citecolor=blue, urlcolor=blue, pdfpagemode=UseOutlines, pdfauthor={H. Asorey},
pdftitle={IPAC 2017 - Guía 01}]{hyperref}
\usepackage[adobe-utopia]{mathdesign}

\hoffset -1.23cm
\textwidth 16.5cm
\voffset -2.0cm
\textheight 26.0cm

%----------------------------------------------------------------
\begin{document}
\title{
{\normalsize{Universidad Nacional de Río Negro - Profesorado de Física}}\\
Introducción a Partículas, Astrofísica y Cosmología \\ Guía 01 - Relatividad\\}
\author{Asorey}
\date{2017}
\maketitle

\begin{enumerate}
	\setcounter{enumi}{0}      %% Offset en numero de problema
	\item Usted se encuentra sobre una plataforma de ferrocarril observando el
		paso de un tren que se mueve a velocidad $v$ en la dirección $+x$. Se
		enciende una luz en el interior de uno de los vagones del tren, y poco
		después se apaga. a) ¿Quién puede medir el intervalo de tiempo propio
		de la duración de la luz: usted o un pasajero a bordo del tren? b)
		¿Quién puede medir la longitud propia del vagón: usted o un pasajero a
		bordo del tren? c) ¿Quién puede medir la longitud propia de un letrero
		fijo en un poste de la plataforma de ferrocarril: usted o un pasajero a
		bordo del tren?
	\item La expectativa de vida de un ser humano es de aproximadamente 75
		años. ¿Significa eso que nunca podremos alcanzar estrellas más allá de
		75 años luz? Justifique. 
	\item Hemos visto que según la teoría de la relatividad, el límite superior
		para la velocidad de una partículas masiva es menor $u < c$. ¿Existen
		entonces límites superiores a la energía y a la cantidad de movimiento
		de una partícula?
	\item Explique la llamada ``paradoja de los gemelos'' y luego diga porque
		no es en realidad una paradoja lógica.
	\item Recordando que la definición relativista de la cantidad de movimiento
		es $\vec{p}=m \gamma \vec{v}$, y que la generalización de la segunda
		ley de Newton es $\vec{F}=d\vec{p}/dt$, obtenga una expresión
		equivalente para la fuerza en función de la aceleración. Suponga que la
		dirección de la velocidad $v$ es $+x$. (ayuda: debe usar la regla de la
		cadena).   
	\item ¿Con qué velocidad debe viajar un cohete en relación con la Tierra de
		manera que el tiempo en el cohete disminuya a la mitad de su tasa al
		ser medida por los observadores de la Tierra?
	\item El muón negativo ($\mu^-$) es una partícula inestable con una vida
		media de $2.2\times 10^{-6}$\,s (medida en el marco en reposo del
		muón). a) Si el muón se mueve con velocidad cercana a la velocidad de
		la luz respecto al laboratorio, se mide en este una vida media de
		$3.1\times 10^{-5}$\,s. Calcule la velocidad del muón. b) ¿Qué
		distancia, medida en el laboratorio, recorre el muón durante su vida
		media? c) ¿cuánto tiempo podría vivir un muón, medido en nuestro marco
		de referencia, que se desplaza a una velocidad de $0.99999c$.
	\item Un regla de dos metros de longitud (medida en el marco propio) pasa
		junto a usted desplazándose a gran velocidad en la dirección $+x$. El
		eje longitudinal de la regla coincide con el eje $x$. Usted mide la
		longitud de la regla en movimiento y encuentra que esta es de un metro,
		al compararla con una regla de un metro que se encuentra en reposo
		junto a usted. ¿Cuál es la velocidad de la regla de dos metros respecto
		a usted?
	\item Una nave espacial extraterrestre se aproxima a la Tierra con
		velocidad $u_1=0.8c$. Se envía desde la Tierra una sonda a recibirla,
		con una velocidad $u_2=0.5c$. a) ¿Cuál es la velocidad de la sonda
		vista desde la nave? b) Si desde la sonda se envía un rayo láser, ¿cuál
		es la velocidad del láser medida desde la sonda? ¿y desde la nave? ¿y
		desde la Tierra?
	\item Para que velocidad $v$ los efectos relativistas son del a) ¿$1\%$?;
		b) $10\%$; c) $100\%$.
	\item Vista desde el laboratorio, una partícula de masa $m_1=10$\,kg se con
		una velocidad $u_1=4/5 c$ en rumbo de colisión con una partícula de
		masa $m_2=15$\,kg que se mueve con una velocidad $u_2=-2/3c$. a)
		Suponiendo un choque totalmente inelástico (ambas partículas quedan
		unidas), y utilizando la conservación de la cantidad de movimiento
		relativista, calcule la velocidad, la energía total, la energía
		cinética y la masa de la partícula resultante. b) Repita sus cálculos
		pero usando la aproximación Newtoniana. 
	\item ¿Cuál es la velocidad de un electrón ($m=0.511$\,MeV/$c^2$) si su
		energía cinética es igual a: a) su energía en reposo? b) tres veces su
		energía en reposo?  c) treinta veces su energía en reposo. En cada
		caso, calcule además la cantidad de movimiento y la energía total. 
	\item Calcule la velocidad (en términos de $c$), la cantidad de movimiento
		y la energía cinética para un protón ($m_p=938.3$\,MeV/$c^2$) del LHC
		cuya energía total es $E=7$\,TeV.
	\item La creación de pares es un proceso mediante el cual, en presencia de
		un átomo, un fotón se convierte en un par electrón-positrón, $\gamma
		\to e^- e^+$. Sabiendo que la masa de estas partículas es
		$511$\,keV/c$^2$, calcule la energía, la longitud de onda y la
		frecuencia del fotón original.
	\item a) Calcule la luminosidad Solar sabiendo que cada segundo
		aproximadamente cuatro millones de toneladas de masa se convierten en
		energía. b) Si cada reacción de fusión libera $25.7$\,MeV de energía,
		calcule el número de reacciones por segundo. c) Calcule la diferencia
		de masa entre los reactivos y los productos de la reacción nuclear.  
\end{enumerate}
\end{document}
%%%%
