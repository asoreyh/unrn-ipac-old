\documentclass[a4paper,12pt]{article}
\usepackage[spanish]{babel}
\hyphenation{co-rres-pon-dien-te}
%\usepackage[latin1]{inputenc}
\usepackage[utf8]{inputenc}
\usepackage[T1]{fontenc}
\usepackage{graphicx}
\usepackage[pdftex,colorlinks=true, pdfstartview=FitH, linkcolor=blue,
citecolor=blue, urlcolor=blue, pdfpagemode=UseOutlines, pdfauthor={H. Asorey},
pdftitle={IPAC 2017 - Guía 02}]{hyperref}
\usepackage[adobe-utopia]{mathdesign}

\hoffset -1.23cm
\textwidth 16.5cm
\voffset -2.0cm
\textheight 26.0cm

%----------------------------------------------------------------
\begin{document}
\title{
{\normalsize{Universidad Nacional de Río Negro - Profesorado de Física}}\\
Introducción a Partículas, Astrofísica y Cosmología \\ Guía 02 - Astrofísica\\}
\author{Asorey}
\date{2017}
\maketitle

\begin{enumerate}
\setcounter{enumi}{15}      %% Offset en numero de problema
\item{\bf{Magnitud aparente}}.
	
	\begin{enumerate}
		\item Calcule la relación entre el brillo de dos estrellas de
			magnitudes aparentes $m_1=1.3$ y $m_2=4.9$.
		\item ¿Cuál es la más brillante?
		\item ¿Cuál sería la magnitud aparente de la segunda estrella si fuera
			10 veces más brillante que la primera?
		\item Las magnitudes aparentes del Sol y de la Luna son:
			$m_\odot=-26.73$ y $m_{\mathrm{Luna}}=-12.6$. Verifique que el Sol
			es 449000 veces más brillante que la Luna.
	\end{enumerate}

\item{\bf{Magnitud absoluta}}

	Para las estrellas identificadas en la foto de la página 64/71 de la clase
		U02C02, obtenga de una tabla las magnitudes absolutas y luego calcule
		las distancias a la Tierra.  

\item{\bf{Estrellas}}
	
	Calcule la luminosidad de Betelgeuse ($M=-5.6$) y de Rigel ($M=-7.0$),
		sabiendo que la magnitud absoluta del Sol es $M=4.83$ y su luminosidad
		$L_\odot = 3.85\times10^{26}$\,W.

\item{\bf{Colores}}

	Utilice la ley de Wien para determinar $\lambda_{\max}$, la
		correspondientes frecuencia $f_{\max}$, y el color aproximado que
		espera para cada una de las siguientes estrellas:\\
		\begin{tabular}{ l | r }
			Nombre & $T$ [K] \\
			\hline
			Sol & $5777$ \\
			Mintaka & $31000$ \\
			Betelgeuse & $3400$ \\
			Sirio A & $9540$ \\
			Rigel & $11000$ \\
			$\eta$-Carinae & $39000$
		\end{tabular}
	
\item{\bf{Temperatura orbital}}
	
\begin{enumerate}
	\item Sabiendo que la distancia del Sol al planeta Marte ($R=3400$\,km) es
		$230$ millones de kilómetros, calcule la temperatura orbital en Marte.
		¿Se encuentra dentro de la zona de habitabilidad Solar?
	\item Calcule la temperatura orbital para un planeta idéntico a la Tierra
		que se encuentra a una distancia $d=1$\,U.A. para estrellas de los
		siguientes tipos espectrales: G5, M20, A5 y B5 (use los espectros de la
		clase U02C03 para asociar una temperatura).
\end{enumerate}

\item {\bf{Diagrama H-R}}

	Busque en catálogos estelares (por ejemplo el catálogo Herschel disponible
		en línea) las características espectrales y de temperatura de las
		siguientes estrellas: Canopus, Antares, Sol, próxima Centauri, Alfa
		Centauri A, Sirio A, Mintaka A, Rigel, Betelgeuse, Procyon, Polaris,
		Spica y Daneb. Luego ubíquelas en un diagrama de HR.

\item{\bf{Betelgeuse y Rigel}}

Betelgeuse($\alpha$-Ori) y Rigel ($\beta$-Ori) son las dos estrellas más
brillantes de la constelación de Orión. Sus posiciones se conocen con excelente
precisión, habiéndose medido un paralaje de $5.07\times10^{-3}$\,arcs para
Betelgeuse y $4.22\times10^{-3}$\,arcseg para Rigel. Utilizando un bolómetro
en órbita, ha sido posible medir los flujos de energía en la Tierra:
$\mathcal{F}_{\mathrm{Betelgeuse}} = 8.6845$\,W\,m$^{-2}$ y
$\mathcal{F}_{\mathrm{Rigel}} = 3.7819$\,W\,m$^{-2}$.
\begin{enumerate}
	\item Calcule la distancia de la Tierra a estas dos estrellas, medidas en
		m, años-luz y parsecs (recuerde que si el paralaje es $1$\,arcs la
		estrella se encuentra a $1$\,parsec de distancia, y la relación es
		inversamente proporcional).
	\item Calcule las luminosidades de Betelgeuse y Rigel.  Expresarlas en
		unidades de $L_\odot$ y en W.
	\item Calcule las masas de Betelgeuse y Rigel ($M_\odot =
		1.899\times10^{30}$kg).
	\item Utilice las temperaturas de las estrellas para estimar los radios de
		las mismas. 
	\item ¿Dentro de que clasificación espectral las colocaría? ¿En que
		posición del diagrama H-R las ubicaría? Justifique.
	\item Calcule los radios mínimos y máximos de la zona de habitabilidad de
		cada estrella.
\end{enumerate}

\item{\bf{Observación astronómica}}

Durante el invierno, mirando hacia el Este y a media altura antes de la
medianoche es posible observar la constelación de Scorpio.  La estrella más
brillante (Antares) se encuentra a 600 años luz de la Tierra.  Sabiendo que
tiene el mismo color que Betelgeuse y que su masa es $M=15.5 M_\odot$, calcule
la Luminosidad y el radio de Antares. Luego determine el tamaño mínimo y máximo
de la zona de habitabilidad.

\item{\bf{Supernova supernueva}}

Cuando una estrella se convierte en supernova, hasta el 1\% de su masa se
libera en forma de energía. De esta energía, el 99\% se libera en forma de
neutrinos y el resto como radiación electromagnética. Imaginemos que Canopus
($\alpha$-Car, F0, $M=8.5\ M_\odot$, $d=310$ años-luz) se convierte en
supernova.
\begin{enumerate}
	\item Calcule la cantidad de energía liberada como neutrinos
		(indetectable).
	\item Calcule el flujo de energía electromagnética que se medirá en la
		Tierra.  Compare este valor con la constante solar,
		$\mathcal{F}_\odot=1400$\,W\,m$^{-2}$.
	\item El objeto resultante será una estrella de neutrones, con un radio
		aproximado de $R=20$\,km. Calcule la densidad, el valor de $g$ y la
		velocidad de escape $v_e$ sobre la superficie de la estrella de
		neutrones.
	\item Calcule el radio de Schwarzschild de Canopus. Compárelo con el
		obtenido para el Sol.
\end{enumerate}

\item{\bf{Producción de energía}}
	
	La masa de un núcleo es menor que la suma de las masas de los protones y
	neutrones que lo componen. Esto se debe a la contribución negativa de la
	energía de unión, que según la relación $E=mc^2$ corresponde a una masa.
	Esa diferencia se denomina {\emph{defecto de masa}}:
	\[ \Delta m = N m_n + Z m_p - m\] donde:
	\begin{itemize}
		\item $m$ es la masa del núcleo
		\item $N$ es el número de neutrones (por ende $N=A-Z$, dónde $A$ es el
			número másico)
		\item $Z$ es el número atómico (igual al número de protones)
		\item $m_p = 938.3$\,MeV/c$^2$ y $m_n = 939.6$\,MeV/c$^2$ son las masas
			del protón y del neutrón respectivamente.
	\end{itemize}
	
	En este contexto, la energía de ligadura por nucleón queda dada por: \[ B =
	\frac{\Delta m c^2}{A}.\] Calcule el defecto de masa y la energía de
	ligadura por nucleón de los siguientes átomos:
	\begin{enumerate}
		\item $^4_2$He (Helio-4, $m=3728.4$\,MeV/c$^2$).
		\item $^{56}_{26}$Fe (Hierro-56, $m=52103$\,MeV/c$^2$).
		\item $^{208}_{82}$Pb (Plomo-208, $m=193729$\,MeV/c$^2$).
		\item $^{40}_{20}$Ca (Calcio-40, $m=37225$\,MeV/c$^2$).
		\item $^{41}_{20}$Ca (Calcio-41, $m=38156$\,MeV/c$^2$).
	\end{enumerate}

\item {\bf{Fusión}}

	Una de las reacciones nucleares que ocurren en los núcleos estelares es la
	denominada cadena p-p, según la cual se produce un núcleo de Helio-4 a
	partir de 4 protones. Esta reacción puede resumirse como: \[4 ^1\mathrm{H}
	\to ^4\mathrm{He} + 2 e^- + 2 \nu_e + Q.\] Sabiendo que las masa total
	inicial es $m_i=4 m_p$ y que la masa final de los productos es
	$m_f=9.91\times10^{-27}$\,kg, calcule: 
	\begin{enumerate}
		\item la energía liberada por cada reacción;
		\item la cantidad de reacciones por segundo que deben producirse para
			explicar la luminosidad del Sol; 
		\item la cantidad de masa que el Sol pierde cada segundo; y 
		\item la cantidad de hidrógeno que es convertido a Helio cada segundo. 
	\end{enumerate}

\item {\bf{Kepler 1}}
	
	Recordemos la primera ley de Kepler: ``{\emph{Todos los planetas se
	desplazan alrededor del Sol describiendo órbitas elípticas, estando el Sol
	situado en uno de sus focos}}''. Utilizando los valores del afelio,
	perihelio y excentricidad de Mercurio, Venus, la Tierra, Urano y Plutón,
	calcule para cada uno de ellos lo siguiente:
	\begin{enumerate}
		\item Los valores de $a$ y $b$ para cada una de las órbitas;
		\item La distancia desde el ``{\emph{otro foco}}'' al Sol.
	\end{enumerate}

\item {\bf{Satélites}}

	\begin{enumerate}
		\item A partir de la expresión para la velocidad orbital de una órbita
			circular, \[v_O = \frac{2 \pi r}{t},\] usando la tercera ley de
			Kepler demuestre que para el caso circular la velocidad orbital
			vale: \[v_O = \sqrt{\frac{GM}{r}}. \] dónde $G$ es la constante de
			gravitación universal, $M$ es la masa del cuerpo central, $r$ es el
			radio de la órbita y $T$ es el periodo orbital.
		\item Calcule el radio $r$ para la órbita de un satélite
			geoestacionario ($T=24$\,horas).
		\item Calcule la velocidad orbital de la estación espacial
			internacional, que se encuentra a una altura media de $330$\,km.
			Luego determine el tiempo requerido para completar una órbita.
	\end{enumerate}

\item {\bf{Cómeta}}
	
	Un nuevo cometa de masa $m=10^{12}$\,kg fue descubierto en el sistema
	solar.  Luego de algunas mediciones, se supo que su órbita es elíptica y el
	perihelio está situado a sólo $10^6$ km del Sol.

\begin{enumerate}
	\item Calcule la distancia al Sol del afelio sabiendo que el período es de
		10 años.
	\item ¿Cuáles es el valor de la energía potencial en el perihelio y en el
		afelio?
	\item Usando la segunda ley de Kepler, calcule la relación entre las
		energías cinéticas en el afelio y en el perihelio (ayuda: suponga que
		las áreas barridas son triangulares, $A=\frac{1}{2} b \times h$).
\end{enumerate}

\item {\bf{Otras Tierras}}

	Imagine que un planeta de masa $m=M_\mathrm{Tierra}$ orbita en torno a una
	estrella de masa $M=6.5 \times 10^{30}$\,kg a una distancia media
	$r=4.5\times10^8$\,km. Usando las leyes de Kepler, calcule el tiempo que
	requiere el planeta para completar una órbita completa y su velocidad
	orbital media.

\end{enumerate}
\end{document}
%%%%
